\def\notes{1}
\documentclass[11pt]{article}
%\usepackage[T1]{fontenc}
%\usepackage[letterpaper]{geometry}
%\geometry{verbose,tmargin=3cm,bmargin=3cm,lmargin=3cm,rmargin=3cm}
%\usepackage{verbatim}
\usepackage{amsfonts,amsmath,amssymb,amsthm}
%\usepackage{setspace}
\usepackage{xspace}%,enumitem}
%\usepackage{times}
\usepackage{fullpage,array}
%\usepackage{hyperref}
%\doublespacing
\usepackage{color}
\usepackage{numdef}
\usepackage{enumitem}
\usepackage{amsopn}
%\usepackage{hyperref} 
\usepackage{mathrsfs}
\usepackage{float}
\restylefloat{table}

\providecommand{\sqbinom}{\genfrac{\lbrack}{\rbrack}{0pt}{}}
\DeclareMathOperator{\spn}{span}

\newif\ifdraft
%\drafttrue
\draftfalse
%%%%%%%%%%%%%%%%%%%%%%%%%%%%% Textclass specific LaTeX commands.
%\numberwithin{equation}{section} %% Comment out for sequentially-numbered
\numberwithin{figure}{section} %% Comment out for sequentially-numbered
\newtheorem{thm}{Theorem}[section]
\newtheorem{conjecture}[thm]{Conjecture}
\newtheorem{definition}[thm]{Definition}
\newtheorem{dfn}[thm]{Definition}
\newtheorem{lemma}[thm]{Lemma}
\newtheorem{remark}[thm]{Remark}
\newtheorem{proposition}[thm]{Proposition}
\newtheorem{corollary}[thm]{Corollary}
\newtheorem{claim}[thm]{Claim}
\newtheorem{fact}[thm]{Fact}
\newtheorem{openprob}[thm]{Open Problem}
\newtheorem{remk}[thm]{Remark}
\newtheorem{example}[thm]{Example}
\newtheorem{apdxlemma}{Lemma}
\newcommand{\question}[1]{{\sf [#1]\marginpar{?}} }
%\usepackage{babel}
\usepackage{tikz}

\makeatletter
\newcommand*{\circled}{\@ifstar\circledstar\circlednostar}
\makeatother

\newcommand*\circledstar[1]{%
  \tikz[baseline=(C.base)]
    \node[%
      fill,
      circle,
      minimum size=1.35em,
      text=white,
      font=\sffamily,
      inner sep=0.5pt
    ](C) {#1};%
}
\newcommand*\circlednostar[1]{%
  \tikz[baseline=(C.base)]
    \node[%
      draw,
      circle,
      minimum size=1.35em,
      font=\sffamily,
      inner sep=0.5pt
    ](C) {#1};%
}

\providecommand{\bd}[1]{\circled{#1}}


%Eli's macros
\newcommand{\E}{{\mathbb{E}}}
%\newcommand{\N}{{\mathbb{N}}}
\newcommand{\R}{{\mathbb{R}}}
\newcommand{\calF}{{\cal{F}}}
%\newcommand{\inp}[1]{\langle{#1}\rangle}
\newcommand{\disp}[1]{D_{#1}}
%\newcommand{\spn}[1]{{\rm{span}}\left(#1\right)}
%\newcommand{\supp}{{\rm{supp}}}
\newcommand{\agree}{{\sf{agree}}}
\newcommand{\RS}{{\sf{RS}}}
\newcommand{\aRS}{{\sf{aRS}}}
\newcommand{\rrs}{R_{\aRS}}
\newcommand{\VRS}{V_\RS}
\newcommand{\x}{\ensuremath{\mathbf{x}}\xspace}
\renewcommand{\a}{\ensuremath{\mathbf{a}}\xspace}

\newcommand{\X}{\ensuremath{\mathbf{X}}\xspace}
\newcommand{\Y}{\ensuremath{\mathbf{Y}}\xspace}
\newcommand{\y}{\ensuremath{\mathbf{y}}\xspace}

\newcommand{\set}[1]{\ensuremath{\left\{#1\right\}}\xspace}
\newcommand{\angles}[1]{\langle{#1}\rangle}
\renewcommand{\span}{\ensuremath{\mathsf{span}}\xspace}
\newcommand{\condset}[2]{\set{#1 \mid #2 }}
\newcommand{\zo}{\set{0,1}}
\newcommand{\zon}{{\zo^n}}
\newcommand{\zom}{{\zo^m}}
\newcommand{\zok}{{\zo^}}
\newcommand{\eps}{\epsilon}
\newcommand{\e}{\eps}
%\newcommand{\d}{\delta}
\newcommand{\poly}{\ensuremath{\mathrm{poly}(\lambda)}\xspace}
\newcommand{\itone}{{\it{(i)}}\xspace}
\newcommand{\ittwo}{{\it{(ii)}}\xspace}
\newcommand{\itthree}{{\it{(iii)}}\xspace}
\newcommand{\itfour}{{\it{(iv)}}\xspace}
\newcommand{\cc}{{\rm{CC}}}
\newcommand{\rnk}{{\rm{rank}}}
\newcommand{\T}{T}
\newcommand{\I}{{\mathbb{I}}}
\ifdraft
\newcommand{\ariel}[1]{{\color{blue}{\textit{#1 --- ariel gabizon}}}}
\else
\newcommand{\ariel}[1]{}
\fi
\providecommand{\improvement}[1]{{\color{red} \textbf{#1}}}
\title{%
Circuit for batching Groth16 proofs}
\date{\today}
\author{}
\begin{document}
\maketitle
 \mathchardef\mhyphen="2D

%\num\newcommand{\G1}{\ensuremath{{\mathbb G}_1}\xspace}
\newcommand{\G}{\ensuremath{{\mathbb G}}\xspace}
\newcommand{\Gstar}{\ensuremath{{\mathbb G}^*}\xspace}

%\num\newcommand{\G2}{\ensuremath{{\mathbb G}_2}\xspace}
%\num\newcommand{\G11}{\ensuremath{\G1\setminus \set{0} }\xspace}
%\num\newcommand{\G21}{\ensuremath{\G2\setminus \set{0} }\xspace}
\newcommand{\grouppair}{\ensuremath{G^*}\xspace}

\newcommand{\Gt}{\ensuremath{{\mathbb G}_t}\xspace}
\newcommand{\F}{\ensuremath{\mathbb F}\xspace}
\newcommand{\Fstar}{\ensuremath{\mathbb F^*}\xspace}

\newcommand{\help}[1]{$#1$-helper\xspace}
\newcommand{\randompair}[1]{\ensuremath{\mathsf{randomPair}(#1)}\xspace}
\newcommand{\pair}[1]{$#1$-pair\xspace}
\newcommand{\pairs}[1]{$#1$-pairs\xspace}

\newcommand{\pairone}[1]{\G1-$#1$-pair\xspace}
\newcommand{\pairtwo}[1]{\G2-$#1$-pair\xspace}
\newcommand{\sameratio}[2]{\ensuremath{\mathsf{SameRatio}(#1,#2)}\xspace}
\newcommand{\vecc}[2]{\ensuremath{(#1)_{#2}}\xspace}
\newcommand{\players}{\ensuremath{[n]}\xspace}
\newcommand{\adv}{\ensuremath{\mathcal A}\xspace}
\newcommand{\advprime}{\ensuremath{A'}\xspace}
\newcommand{\extprime}{\ensuremath{E'}\xspace}
\newcommand{\advrand}{\ensuremath{\mathsf{rand}_{\adv}}\xspace}

\newcommand{\ci}{\ensuremath{\mathrm{CI}}\xspace}
\newcommand{\pairvec}[1]{$#1$-vector\xspace}
\newcommand{\Fq}{\ensuremath{\mathbb{F}_q}\xspace}
\newcommand{\randpair}[1]{\ensuremath{\mathsf{rp}_{#1}}\xspace}
\newcommand{\randpairone}[1]{\ensuremath{\mathsf{rp}_{#1}^{1}}\xspace}
\newcommand{\abase}{\ensuremath{A_{\mathrm{\mathbf{0}}}}\xspace}
\newcommand{\bbase}{\ensuremath{B_{\mathrm{\mathbf{0}}}}\xspace}
\newcommand{\cbase}{\ensuremath{C_{\mathrm{\mathbf{0}}}}\xspace}

\newcommand{\amid}{\ensuremath{A_{\mathrm{mid}}}\xspace}
\newcommand{\bmid}{\ensuremath{B_{\mathrm{mid}}}\xspace}
\newcommand{\cmid}{\ensuremath{C_{\mathrm{mid}}}\xspace}

\newcommand{\negl}{\ensuremath{\mathsf{negl}(\lambda)}\xspace}
\newcommand{\randpairtwo}[1]{\ensuremath{\mathsf{rp_{#1}^2}}\xspace}%the randpair in G2
\newcommand{\nilp}{\ensuremath{\mathscr N}\xspace}
\newcommand{\snark}{\ensuremath{\mathscr S}\xspace}
\newcommand{\groupgen}{\ensuremath{\mathscr G}\xspace}
\newcommand{\qap}{\ensuremath{\mathscr Q}\xspace}

\newcommand{\rej}{\ensuremath{\mathsf{rej}}\xspace}
\newcommand{\acc}{\ensuremath{\mathsf{acc}}\xspace}
\newcommand{\sha}[1]{\ensuremath{\mathsf{COMMIT}(#1)}\xspace}
 \newcommand{\shaa}{\ensuremath{\mathsf{COMMIT}}\xspace}
 \newcommand{\comm}[1]{\ensuremath{\mathsf{comm}_{#1}}\xspace}
 \newcommand{\defeq}{:=}

\newcommand{\A}{\ensuremath{\vec{A}}\xspace}
\newcommand{\B}{\ensuremath{\vec{B}}\xspace}
\newcommand{\C}{\ensuremath{\vec{C}}\xspace}
\newcommand{\Btwo}{\ensuremath{\vec{B_2}}\xspace}
\newcommand{\treevecsimp}{\ensuremath{(\tau,\rho_A,\rho_A \rho_B,\rho_A\alpha_A,\rho_A\rho_B\alpha_B, \rho_A\rho_B\alpha_C,\beta,\beta\gamma)}\xspace}% The sets of elements used in simplifed relation tree in main text body
\newcommand{\rcptc}{random-coefficient subprotocol\xspace}
\newcommand{\rcptcparams}[2]{\ensuremath{\mathrm{RCPC}(#1,#2)}\xspace}
\newcommand{\verifyrcptcparams}[2]{\ensuremath{\mathrm{\mathsf{verify}RCPC}(#1,#2)}\xspace}
\newcommand{\randadv}{\ensuremath{\mathsf{rand}_{\adv}}\xspace}
 \num\newcommand{\ex1}[1]{\ensuremath{ #1\cdot g_1}\xspace}
 \num\newcommand{\ex2}[1]{\ensuremath{#1\cdot g_2}\xspace}
 \newcommand{\pr}{\mathrm{Pr}}
 \newcommand{\powervec}[2]{\ensuremath{(1,#1,#1^{2},\ldots,#1^{#2})}\xspace}
\num\newcommand{\out1}[1]{\ensuremath{\ex1{\powervec{#1}{d}}}\xspace}
\num\newcommand{\out2}[1]{\ensuremath{\ex2{\powervec{#1}{d}}}\xspace}
 \newcommand{\nizk}[2]{\ensuremath{\mathrm{NIZK}(#1,#2)}\xspace}% #2 is the hash concatenation input
 \newcommand{\verifynizk}[3]{\ensuremath{\mathrm{VERIFY\mhyphen NIZK}(#1,#2,#3)}\xspace}
\newcommand{\protver}{protocol verifier\xspace} 
\newcommand{\mulgroup}{\ensuremath{\F^*}\xspace}
\newcommand{\lag}[1]{\ensuremath{L_{#1}}\xspace} 
\newcommand{\sett}[2]{\ensuremath{\set{#1}_{#2}}\xspace}
\newcommand{\omegaprod}{\ensuremath{\alpha_{\omega}}\xspace}
\newcommand{\lagvec}[1]{\ensuremath{\mathrm{LAG}_{#1}}\xspace}
\newcommand{\trapdoor}{\ensuremath{r}}
\newcommand{\trapdoorext}{\ensuremath{r_{\mathrm{ext}}}\xspace}
\newcommand{\trapdoorsim}{\ensuremath{r_{\mathrm{sim}}}\xspace}

\num\newcommand{\enc1}[1]{\ensuremath{\left[#1\right]_1}\xspace}
\num\newcommand{\enc2}[1]{\ensuremath{\left[#1\right]_2}\xspace}
\num\newcommand{\enc3}[1]{\ensuremath{\left[#1\right]_t}\xspace}
\newcommand{\gen}{\ensuremath{\mathsf{Gen}}\xspace}
\newcommand{\prv}{\ensuremath{\mathrm{P}}\xspace}
\newcommand{\simprv}{\ensuremath{\mathrm{P^{sim}}}\xspace}

\num\newcommand{\enc0}[1]{\ensuremath{\left[#1\right]}\xspace}
%\num\newcommand{\G0}{\ensuremath{\mathbf{G}}\xspace}
\newcommand{\GG}{\ensuremath{\mathbf{G^*}}\xspace}  % would have liked to call this G01 but problem with name
\num\newcommand{\g0}{\ensuremath{\mathbf{g}}\xspace}
\newcommand{\inst}{\ensuremath{\phi}\xspace}
\newcommand{\inp}{\ensuremath{x}\xspace}
\newcommand{\wit}{\ensuremath{\omega}\xspace}
\newcommand{\ver}{\ensuremath{\mathsf{V}}\xspace}
\newcommand{\per}{\ensuremath{\mathsf{P}}\xspace}
\newcommand{\rel}{\ensuremath{\mathcal{R}}\xspace}
\newcommand{\prf}{\ensuremath{\pi}\xspace}
\newcommand{\ext}{\ensuremath{E}\xspace}
\newcommand{\params}{\ensuremath{\mathsf{params}_{\inst}}\xspace}
\newcommand{\protparams}{\ensuremath{\mathsf{params}_{\inst}^\advv}\xspace}
\num\newcommand{\p1}{\ensuremath{P_1}\xspace}
\newcommand{\advv}{\ensuremath{ {\mathcal A}^{\mathbf{*}}}\xspace} % the adversary that uses protocol adversary as black box
\newcommand{\crs}{\ensuremath{\sigma}\xspace}
%\num\newcommand{\crs1}{\ensuremath{\mathrm{\sigma}_1}\xspace}
%\num\newcommand{\crs2}{\ensuremath{\mathrm{\sigma}_2}\xspace}

\renewcommand{\sim}{\ensuremath{\mathsf{sim}}\xspace}%the distribution of messages when \advv simulates message of \p1
\newcommand{\real}{\ensuremath{\mathsf{real}}\xspace}%the distribution of messages when \p1 is honest and \adv controls rest of players
 \newcommand{\koevec}[2]{\ensuremath{(1,#1,\ldots,#1^{#2},\alpha,\alpha #1,\ldots,\alpha #1^{#2})}\xspace}
\newcommand{\mida}{\ensuremath{A_{\mathrm{mid}}}\xspace}
\newcommand{\midb}{\ensuremath{B_{\mathrm{mid}}}\xspace}
\newcommand{\midc}{\ensuremath{C_{\mathrm{mid}}}\xspace}
\newcommand{\chal}{\ensuremath{\mathsf{challenge}}\xspace}
\newcommand{\attackparams}{\ensuremath{\mathsf{params^{pin}}}\xspace}
\newcommand{\pk}{\ensuremath{\mathsf{pk}}\xspace}
\newcommand{\attackdist}[2]{\ensuremath{AD_{#1}}\xspace}
\renewcommand{\neg}{\ensuremath{\mathsf{negl}(\lambda)}\xspace}
\newcommand{\ro}{\ensuremath{{\mathcal R}}\xspace}
\newcommand{\elements}[1]{\ensuremath{\mathsf{elements}_{#1}}\xspace}
 \num\newcommand{\elmpowers1}[1]{\ensuremath{\mathrm{\mathsf{e}}^1_{#1}}\xspace}
 \num\newcommand{\elmpowers2}[1]{\ensuremath{\mathrm{\mathsf{e}}^2_{#1}}\xspace}
\newcommand{\elempowrs}[1]{\ensuremath{\mathsf{e}_{#1}}\xspace}
 \newcommand{\secrets}{\ensuremath{\mathsf{secrets}}\xspace}
 \newcommand{\polysofdeg}[1]{\ensuremath{\F_{< #1}[X]}\xspace}
\newcommand{\bctv}{\ensuremath{\mathsf{BCTV}}\xspace}
\newcommand{\PI}{\ensuremath{\mathsf{PI}}\xspace}
\newcommand{\ML}{\ensuremath{\mathsf{ML}}\xspace}
\newcommand{\FE}{\ensuremath{\mathsf{FE}}\xspace}
\newcommand{\PIb}{\ensuremath{\mathsf{PI_B}}\xspace}
\newcommand{\PIc}{\ensuremath{\mat?hsf{PI_C}}\xspace}
\newcommand{\dl}[1]{\ensuremath{\widehat{#1}}\xspace}
\newcommand{\pia}{\ensuremath{\mathsf{\pi_{A,i}}}\xspace}
\newcommand{\pib}{\ensuremath{\mathsf{\pi_{B,i}}}\xspace}
\newcommand{\pic}{\ensuremath{\mathsf{\pi_{C,i}}}\xspace}
\newcommand{\ped}{\ensuremath{\mathsf{ped}}\xspace}
\newcommand{\groth}{\ensuremath{\mathsf{Groth16}}\xspace}

For reference, recall the \groth verification equation.

\[e(\pi_A,\pi_B) =\e(\pi_C,\enc2{\delta})\cdot \PI\]
Where \PI is an element in \Gt derived by the verifier from the public input.
Specificially,
\[\PI=\enc3{\alpha\beta+ K_0+\sum_{i=1}^\ell a_i\cdot K_i}\]
where $(a_1,\ldots,a_n)$ is the public input, and $K_i = \beta u_i(x) + \alpha  v_i (x) + w_i(x)$.

We move the second pairing to the LHS to receive
\[e(\pi_A,\pi_B) \cdot \e(-\pi_C,\enc2{\delta})=  \PI\]


We use \ML to denote a miller loop and \FE to denote the final exponetiation.
We assume the verifer and prover both know the set of public inputs and thus the derived values \sett{\PI_i}{i\in [m]}.
The private input for the circuit is a set of $m$ \groth proofs
$S\defeq \sett{\pia,\pib,\pic}{i\in [m]}$

The public inputs of the circuit are 
\begin{enumerate}
 \item $P$ - the alleged pedersen hash of all proof elements.
 \item $F$ - the alleged correct randomized combination of pairings of proof elements.
 \item  $r=\mathsf{Blake(P,(\PI_1,\ldots,\PI_m))}$, interperted as an element of \F.

\end{enumerate}

The verifier has the  corresponding $m$  public input elements $\PI_1,\ldots,\PI_m$, and checks outside of the circuit that
\[F= \prod_{i\in m} \PI_i ^{r^i}\]


The circut computes
\begin{enumerate}

 \item $M\defeq \prod_{i\in [m]} \ML(\pia,\pib).$
 \item $C' \defeq -\sum_{i\in [m]} r^i\cdot \pic$
 \item $C=\ML(C',\enc2{\delta})$
 
\end{enumerate}
 The circuit checks that
 \begin{enumerate}
 \item $F=\FE(M\cdot C)$.
 \item $P=\ped(S)$.
 \end{enumerate}

 \begin{remark}
  It is possible to not do \FE inside the circuit and have the verifier do it outside;
  however since it's only once per batch it might be better to leave it in the circuit.
 \end{remark}

 \bibliographystyle{alpha}
\bibliography{References}
\end{document}
 
